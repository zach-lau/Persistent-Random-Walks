\documentclass{article}
\usepackage{amsmath}
\usepackage{amsthm}
\usepackage{amssymb}
\usepackage[margin=1cm]{geometry}

\title{STAT 547 Proofs}
\author{Zachary Lau, Emma Colllins}
\date{\today}

\begin{document}
\maketitle

\section{Generalized shortcut for TE}
In Birron-Lattes et al, they present a formula for the TE in the case of uniform
level affinities. With $p$ the probability that a tour reaches the top-level, we
have \[ \text{TE} = \frac{p}{2-p}\].

We breifly reproduce this result with application to possibly assymetric
persistent random walks, and by extension non-uniform level affinities.

Claim: the TE for an assymetric random walk is given by \[ \text{TE} =
\frac{p_u}{2-p_d} \]
Where $p_u$ is the probability that a tour reaches the top-level, and $p_d$ is
the probability that starting at the top-level, the $0$-th leve is visited
before the top-level is visited again. I.e. it is the $p_u$ of the mirror
process where the level order is reversed.

\begin{proof}
The number of visits to the top level can be expressed as \[ v \overset{d}{=} RUF \]
where \begin{itemize}
    \item $R$ is Bernouilli with rate $p_u$
    \item $U$ is $\text{Geom1}$ with rate $p_d$
    \item $F$ is a constant that is $2$ for NRST and $1$ for $ST$ to account for
    the fact that $NRST$ deterministically visits the top-level twice.
\end{itemize}
and $R$ and $U$ are independent. With $v$ as the number of visits to the target
during the tour we find
\begin{align*}
\mathbb E[v] &= \frac{p_u}{p_d}F \\
\end{align*}
and 
\begin{align*}
\mathbb E[v^2] &= p_u \left(\frac{1}{p_d^2}+\frac{1-p_d}{p_d^2}\right) F^2 \\
&= \frac{F^2p_u(2-p_d)}{p_d^2}
\end{align*}
Therefore
\begin{align*}
\text{TE} &= \frac{E[v]^2}{E[v^2]} \\
&= \frac{p_u}{2-p_d}
\end{align*}
\end{proof}
In the case $p_u = p_d = p$ we confirm that this reduces to the symmetric case.

\section{Recursive formula for top bottom transition probabilities in NRST}
We present here a recursive formula for the top-bottom transition probabilities.
Denote the forward transition probabilities from $i$ to $i+1$ by $\alpha_i$ and 
reverse transition probabilities from $i+1$ to $i$ of $\alpha'_i$ with rejection
probabilities $\rho_i = 1-\alpha_i$ and $\rho'_i = 1-\alpha'_i$. Denote the 
probability of returning to the base layer from state $(i,-1)$ before returning
to layer $i$ by $p_i$. We claim that $p_i$ can be recursively calculated from
\begin{align*}
    p_1 &= \alpha'_0 \\
    \frac{1}{p_{i+1}} &=
    \frac{\alpha_i}{\alpha'_i}\left(\frac{\rho_i}{\alpha_i}+\frac{1}{p_i}\right)
\end{align*}
\begin{proof}
    From the state $(i+1,-1)$ in order to level $0$ before reaching level $i+1$
    again we require the following to happen \begin{itemize}
        \item The proposal to move to $(i,-1)$ is accepted
        \item From $(i,-1)$ we reach level $0$ before reaching level $i+1$ again
    \end{itemize}
    The first step will happen with probability $\alpha'_i$. For the second
    event we consider the sequence of round-trips starting and ending at $(i,-1)$.
    For each round trip we have to consider the mutually exclusive events \begin{itemize}
        \item A: it reaches $0$
        \item B: it doesn't reach $0$ and it reaches $i+1$
    \end{itemize}
    Any return to level $i$ necessarily starts a new round trip. Thus the first
    event is exactly the event that from $i$ level $0$ is reached before $i$ is
    reached again. This is exactly our recursive event of interest. Thus 
    $P(A)=\alpha'_i p_i$.  Any round trip that does not reach $0$ will reach
    level $i$ exactly once. From this state, it will reach level $i+1$ with
    probability $\alpha_i$. We get $P(B)=(1-p_i)\alpha_i$. These two events are
    mutually exclusive. Therefore the probability that $A$ occurs before $B$
    is \begin{align*}
        p_{i+1} &= \frac{P(A)}{P(A)+P(B)} \\
        &= \frac{\alpha'_i p_i}{p_i+(1-p_i)\alpha_i} \\
        &= \frac{\frac{\alpha'_i}{\alpha_i}}{\frac{\rho_i}{\alpha_i}+\frac{1}{p_i}} \\
        \frac{1}{p_{i+1}} &=
        \frac{\alpha_i}{\alpha'_i}\left(\frac{\rho_i}{\alpha_i}+\frac{1}{p_i}\right)
    \end{align*}
\end{proof}
Importantly when $\alpha_i=\alpha_i'$ we see that
\[ \frac{1}{\rho_{i+1}} = \frac{\rho_i}{\alpha_i}+\frac{1}{\rho_i} \]
And we recover the expression for the symmetric case. 
\begin{align*}
    \frac{1}{p} &= \frac{1}{\alpha_0} + \sum_{i=1}^{N-1} \frac{\rho_i}{\alpha_i} \\
    &= \frac{\alpha_0+\rho_0}{\alpha_0} +  \sum_{i=1}^{N-1} \frac{\rho_i}{\alpha_i} \\
    &= 1+ \sum_{i=0}^{N-1} \frac{\rho_i}{\alpha_i}
\end{align*} 
The same recursion formula can be used to easily find the probability that a
tour reaches the target level using the mirrored chain. These two probabilities
together directly give the tour effectiveness.

\section{Expanded Formula for top bottom transition probabilities in NRST}
Based on the recursive formula above we expand the expression for $p_{n-1}$, the
top-bottom transition probability from the top-level to the base level. Claim:
Denote (for lack of a better letter) by $\gamma_i$ the marginal probability of being
in level $i$ Then
\[ \frac{1}{p_{n}} = \frac{\gamma_n}{\gamma_0} + a_{n}\sum_{i=0}^{n-1}
\frac{1}{\gamma_i}\frac{\rho_i}{\alpha_i} \]

\begin{proof}
    From the mass balance equation we know that $\gamma_i \alpha_i = \gamma_{i+1}\alpha'_i$.
    Plugging this into our recursion above we find 
    \[ \frac{1}{p_{i+1}} 
    = \frac{\gamma_{i+1}}{\gamma_i} \left(\frac{\rho_i}{\alpha_i} + \frac{1}{p_i}\right)\]
    We proceed by induction. In the base case when $n=1$ we have 
    \begin{align*}
        \frac{1}{p_1} &= \frac{1}{\alpha'_0} \\
        &= \frac{\gamma_1}{\gamma_0}\frac{1}{\alpha_0} \\
        &= \frac{\gamma_1}{\gamma_0}\left(1+\frac{\rho_0}{\alpha_0}\right) \\
        &= \frac{\gamma_1}{\gamma_0} + \frac{\gamma_1}{\gamma_0}\frac{\rho_0}{\alpha_0}
    \end{align*}
    Now we assume our induction hypothesis is true for $1 < k \leq n$ and apply
    the recursion
    \begin{align*}
    \frac{1}{p_{k+1}}\
    &= \frac{\gamma_{k+1}}{\gamma_k}\left(\frac{\rho_k}{\alpha_k}+\frac{1}{p_k}\right) \\
    &= \frac{\gamma_{k+1}}{\gamma_k}\frac{\rho_k}{\alpha_k} + 
    \frac{\gamma_{k+1}}{\gamma_k}
    \left(\frac{\gamma_k}{\gamma_0}+\gamma_k\sum_{i=0}^{k-1}\frac{1}{\gamma_i}\frac{\rho_i}{\alpha_i}\right) \\
    &= \frac{\gamma_{k+1}}{\gamma_0} + \gamma_{k+1}\sum_{i=0}^k \frac{1}{\gamma_i}\frac{\rho_i}{\alpha_i}
    \end{align*}
    This completes our inductive step and the proof is complete.
\end{proof}
Again we see that in the case of uniform level affinities this equation reduces
to the familiar \[ \frac{1}{p_n} = 1+\sum_{i=1}^n \frac{\rho_i}{\alpha_i}\] 
In particular, the assymetric case is simply a weighted version of the symmetric
case.

\end{document}