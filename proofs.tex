\documentclass{article}
\usepackage{amsmath}
\usepackage{amsthm}
\usepackage{amssymb}
\usepackage[margin=1cm]{geometry}

\title{STAT 547 Proofs}
\author{Zachary Lau, Emma Colllins}
\date{\today}

\begin{document}
\maketitle

\section{Generalized shortcut for TE}
In Birron-Lattes et al, they present a formula for the TE in the case of uniform
level affinities. With $p$ the probability that a tour reaches the top-level, we
have \[ \text{TE} = \frac{p}{2-p}\].

We breifly reproduce this result with application to possibly assymetric
persistent random walks, and by extension non-uniform level affinities.

Claim: the TE for an assymetric random walk is given by \[ \text{TE} =
\frac{p_u}{2-p_d} \]
Where $p_u$ is the probability that a tour reaches the top-level, and $p_d$ is
the probability that starting at the top-level, the $0$-th leve is visited
before the top-level is visited again. I.e. it is the $p_u$ of the mirror
process where the level order is reversed.

\begin{proof}
The number of visits to the top level can be expressed as \[ v \overset{d}{=} RUF \]
where \begin{itemize}
    \item $R$ is Bernouilli with rate $p_u$
    \item $U$ is $\text{Geom1}$ with rate $p_d$
    \item $F$ is a constant that is $2$ for NRST and $1$ for $ST$ to account for
    the fact that $NRST$ deterministically visits the top-level twice.
\end{itemize}
and $R$ and $U$ are independent. With $v$ as the number of visits to the target
during the tour we find
\begin{align*}
\mathbb E[v] &= \frac{p_u}{p_d}F \\
\end{align*}
and 
\begin{align*}
\mathbb E[v^2] &= p_u \left(\frac{1}{p_d^2}+\frac{1-p_d}{p_d^2}\right) F^2 \\
&= \frac{F^2p_u(2-p_d)}{p_d^2}
\end{align*}
Therefore
\begin{align*}
\text{TE} &= \frac{E[v]^2}{E[v^2]} \\
&= \frac{p_u}{2-p_d}
\end{align*}
\end{proof}
In the case $p_u = p_d = p$ we confirm that this reduces to the symmetric case.

\section{Recursive formula for top bottom transition probabilities in NRST}
We present here a recursive formula for the top-bottom transition probabilities.
Denote the forward transition probabilities from $i$ to $i+1$ by $\alpha_i$ and 
reverse transition probabilities from $i+1$ to $i$ of $\alpha'_i$ with rejection
probabilities $\rho_i = 1-\alpha_i$ and $\rho'_i = 1-\alpha'_i$. Denote the 
probability of returning to the base layer from state $(i,-1)$ before returning
to layer $i$ by $p_i$. We claim that $p_i$ can be recursively calculated from
\begin{align*}
    p_1 &= \alpha'_0 \\
    \frac{1}{p_{i+1}} &=
    \frac{\alpha_i}{\alpha'_i}\left(\frac{\rho_i}{\alpha_i}+\frac{1}{p_i}\right)
\end{align*}
\begin{proof}
    From the state $(i+1,-1)$ in order to level $0$ before reaching level $i+1$
    again we require the following to happen \begin{itemize}
        \item The proposal to move to $(i,-1)$ is accepted
        \item From $(i,-1)$ we reach level $0$ before reaching level $i+1$ again
    \end{itemize}
    The first step will happen with probability $\alpha'_i$. For the second
    event we consider the sequence of round-trips starting and ending at $(i,-1)$.
    For each round trip we have to consider the mutually exclusive events \begin{itemize}
        \item A: it reaches $0$
        \item B: it doesn't reach $0$ and it reaches $i+1$
    \end{itemize}
    Any return to level $i$ necessarily starts a new round trip. Thus the first
    event is exactly the event that from $i$ level $0$ is reached before $i$ is
    reached again. This is exactly our recursive event of interest. Thus 
    $P(A)=\alpha'_i p_i$.  Any round trip that does not reach $0$ will reach
    level $i$ exactly once. From this state, it will reach level $i+1$ with
    probability $\alpha_i$. We get $P(B)=(1-p_i)\alpha_i$. These two events are
    mutually exclusive. Therefore the probability that $A$ occurs before $B$
    is \begin{align*}
        p_{i+1} &= \frac{P(A)}{P(A)+P(B)} \\
        &= \frac{\alpha'_i p_i}{p_i+(1-p_i)\alpha_i} \\
        &= \frac{\frac{\alpha'_i}{\alpha_i}}{\frac{\rho_i}{\alpha_i}+\frac{1}{p_i}} \\
        \frac{1}{p_{i+1}} &=
        \frac{\alpha_i}{\alpha'_i}\left(\frac{\rho_i}{\alpha_i}+\frac{1}{p_i}\right)
    \end{align*}
\end{proof}
Importantly when $\alpha_i=\alpha_i'$ we see that
\[ \frac{1}{\rho_{i+1}} = \frac{\rho_i}{\alpha_i}+\frac{1}{\rho_i} \]
And we recover the expression for the symmetric case. 
\begin{align*}
    \frac{1}{p} &= \frac{1}{\alpha_0} + \sum_{i=1}^{N-1} \frac{\rho_i}{\alpha_i} \\
    &= \frac{\alpha_0+\rho_0}{\alpha_0} +  \sum_{i=1}^{N-1} \frac{\rho_i}{\alpha_i} \\
    &= 1+ \sum_{i=0}^{N-1} \frac{\rho_i}{\alpha_i}
\end{align*} 
The same recursion formula can be used to easily find the probability that a
tour reaches the target level using the mirrored chain. These two probabilities
together directly give the tour effectiveness.

\section{Expanded Formula for top bottom transition probabilities in NRST}
Based on the recursive formula above we expand the expression for $p_{n-1}$, the
top-bottom transition probability from the top-level to the base level. Claim:
Denote (for lack of a better letter) by $\gamma_i$ the marginal probability of being
in level $i$ Then
\[ \frac{1}{p_{n}} = \frac{\gamma_n}{\gamma_0} + \gamma_{n}\sum_{i=0}^{n-1}
\frac{1}{\gamma_i}\frac{\rho_i}{\alpha_i} \]

\begin{proof}
    From the mass balance equation we know that $\gamma_i \alpha_i = \gamma_{i+1}\alpha'_i$.
    Plugging this into our recursion above we find 
    \[ \frac{1}{p_{i+1}} 
    = \frac{\gamma_{i+1}}{\gamma_i} \left(\frac{\rho_i}{\alpha_i} + \frac{1}{p_i}\right)\]
    We proceed by induction. In the base case when $n=1$ we have 
    \begin{align*}
        \frac{1}{p_1} &= \frac{1}{\alpha'_0} \\
        &= \frac{\gamma_1}{\gamma_0}\frac{1}{\alpha_0} \\
        &= \frac{\gamma_1}{\gamma_0}\left(1+\frac{\rho_0}{\alpha_0}\right) \\
        &= \frac{\gamma_1}{\gamma_0} + \frac{\gamma_1}{\gamma_0}\frac{\rho_0}{\alpha_0}
    \end{align*}
    Now we assume our induction hypothesis is true for $1 < k \leq n$ and apply
    the recursion
    \begin{align*}
    \frac{1}{p_{k+1}}\
    &= \frac{\gamma_{k+1}}{\gamma_k}\left(\frac{\rho_k}{\alpha_k}+\frac{1}{p_k}\right) \\
    &= \frac{\gamma_{k+1}}{\gamma_k}\frac{\rho_k}{\alpha_k} + 
    \frac{\gamma_{k+1}}{\gamma_k}
    \left(\frac{\gamma_k}{\gamma_0}+\gamma_k\sum_{i=0}^{k-1}\frac{1}{\gamma_i}\frac{\rho_i}{\alpha_i}\right) \\
    &= \frac{\gamma_{k+1}}{\gamma_0} + \gamma_{k+1}\sum_{i=0}^k \frac{1}{\gamma_i}\frac{\rho_i}{\alpha_i}
    \end{align*}
    This completes our inductive step and the proof is complete.
\end{proof}
Again we see that in the case of uniform level affinities this equation reduces
to the familiar \[ \frac{1}{p_n} = 1+\sum_{i=1}^n \frac{\rho_i}{\alpha_i}\] 
In particular, the assymetric case is simply a weighted version of the symmetric
case.

\section{Recursive formula for top-bottom transition probabilities in reversible
simulated tempering}
We now present a recursive formula for the downward transition probabilities in
reversible simulated tempring. In the absence of momentum we model the Markov
chain with the symmetric proposal which proposes to go up with probability $1/2$
and down with probability with $1/2$. The acceptance rates remain the same, thus
we get transition probabilites
\begin{itemize}
    \item $P_{i,i+1} = \frac{\alpha_i}{2}$
    \item $P_{i+1,i} = \frac{\alpha'_i}{2}$
    \item $P_{i,i} = \frac{\rho_i+\rho'_i}{2}$
\end{itemize}
where $i$ indicates the level. Denote by $p_i$ the probability that starting
from level $i$ the bottom level is reached before level $i$ is reached again.
We claim that the $p_i$ can be recursively calculated by
\[ \frac{1}{p_{i+1}} =
\frac{\alpha_i}{\alpha'_i}\left(2+2\frac{\rho_i}{\alpha_i} +
\frac{1}{p_i}\right)\]
And in the base case $\frac{1}{p_1} = 2+2\frac{\rho'_0}{\alpha'_0}$

\begin{proof}
We start with the base case. Any transition from level $1$ which does not visit
level $0$ will necessarily lead to another visit to level $1$ before a visit to
level $0$. Thus from the transition probabilities
\begin{align*}
p_1 &= \frac{\alpha'_0}{2} \\
\frac{1}{p_1} &= \frac{2}{\alpha'_0} \\
&= \frac{2(\alpha'_0+\rho'_0)}{\alpha'_0} \\
&= 2 + 2\frac{\rho'_0}{\alpha'_0}
\end{align*}

Next consider the recursive step. From level $i+1$ to visit the base level
before revisiting $i+1$, we need the following to happen 
\begin{enumerate}
    \item Transition immediatley to level $i$
    \item Transition from level $i$ to the base level before transition back to
    level $i+1$
\end{enumerate}
The first event occurs with probability $\alpha'_i/2$. For the second event we
consdir each round-trip starting at level $i$ and ending at level $i$. One
of two mutually exclusive events may occurs
\begin{enumerate}
    \item $A$: we visit level $0$
    \item $B$: we visit level $i+1$
\end{enumerate}
The probability of the first event is simply $p_i$. The probablity of the second
event is just the transition probability $\frac{\alpha_i}{2}$. Theefore the
probability that $A$ occurs before $B$ is 
\begin{align*}
\frac{P(A)}{P(A)+P(B)} &= \frac{p_i}{\alpha_{i}/2 + p_i}
\end{align*}
These two events must occur sequentially so we find 
\begin{align*}
p_{i+1} &= \frac{\alpha'_i}{2}\left(\frac{p_i}{p_i+\frac{\alpha_i}{2}}\right) \\
&= \frac{\alpha'_i}{\alpha_i}\left(\frac{1}{\frac{2}{\alpha_i}+\frac{1}{p_i}}\right) \\
\frac{1}{p_{i+1}}&= \frac{\alpha_i}{\alpha'_i}\left(\frac{2}{\alpha_i}+\frac{1}{p_i}\right) \\
&= \frac{\alpha_i}{\alpha'_i}\left(2+2\frac{\rho_i}{\alpha_i}+\frac{1}{p_i}\right) \\ 
\end{align*}
\end{proof}

\section{Expanded formula for top-bottom transition probabilites in simulated
tempering}
We can expand the recursive formula to get the full formula given by 
\[ \frac{1}{p_n} = 2\sum_{i=0}^{n-1}\frac{\gamma_n}{\gamma_i} +
2\sum_{i=0}^{n-1}\frac{\gamma_n}{\gamma_i}\frac{\rho_i}{\alpha_i}\]

\begin{proof}
    We prove the above statement by induction. In our base case we take
    $n=1$. In this case noting that $\alpha_0+\rho_0=1$
    \begin{align*}
        \frac{1}{p_1} &= \frac{2}{\alpha'_0} \\
        &= \frac{2\gamma_1}{\gamma_0\alpha_0} \\
        &= 2\frac{\gamma_1}{\gamma_0}+2\frac{\gamma_1}{\gamma_0}\frac{\rho_0}{\alpha_0}
    \end{align*}

    For the inductive step we assume that the inductive hypothesis is true
    for $n=k \geq 1$. Applying the recursive formula
    \begin{align*}
    \frac{1}{p_{k+1}} &= 
    \frac{\gamma_{k+1}}{\gamma_k}\left(2+
    2\frac{\rho_k}{\alpha_k}+
    2\sum_{i=0}^{k-1}\frac{\gamma_k}{\gamma_i} + 
    2\sum_{i=0}^{k-1}\frac{\gamma_k}{\gamma_0}\frac{\rho_i}{\alpha_i}\right) \\
    &= 2\sum_{i=0}^{k} \frac{\gamma_{k+1}}{\gamma_i} +
    2\sum_{i=0}^{k}\frac{\gamma_{k+1}}{\gamma_i}\frac{\rho_i}{\alpha_i}
    \end{align*}
\end{proof}
Notably in the case of uniform level affinities we get
\[ \frac{1}{p_n} = 2(n+1) + 2\sum_{i=0}^k \frac{\rho_i}{\alpha_i} \]
This matches what has previously been shown by Birron-Lattes et al. Note that
due to our choice of indexing we have an $n+1$ where they have an $n$.

\section{Proof that NRST has higher TE than an equivalent ST algorithm
when level affinites are non-uniform}

\end{document}